\documentclass[11pt]{article}
\usepackage{amssymb, amsthm, amsmath}
\usepackage{bm}
\usepackage{graphicx}
\usepackage[authoryear]{natbib}
\usepackage{bm}
\usepackage{verbatim}
\usepackage{lineno}
\usepackage{times}
\usepackage{soul}
\usepackage{color}
\usepackage{enumerate}
\usepackage{setspace}
\usepackage{times}
\usepackage{changepage}
\usepackage{multirow}
\usepackage{booktabs}

\usepackage[left=1in,top=1in,right=1in]{geometry}
\pdfpageheight 11in
\pdfpagewidth 8.5in
\linespread{2.0}
\newcommand{\btheta}{ \mbox{\boldmath $\theta$}}
\newcommand{\bmu}{ \mbox{\boldmath $\mu$}}
\newcommand{\balpha}{ \mbox{\boldmath $\alpha$}}
\newcommand{\bbeta}{ \mbox{\boldmath $\beta$}}
\newcommand{\tbeta}{ \mbox{$\tilde \beta$}}
\newcommand{\bdelta}{ \mbox{\boldmath $\delta$}}
\newcommand{\blambda}{ \mbox{\boldmath $\lambda$}}
\newcommand{\bgamma}{ \mbox{\boldmath $\gamma$}}
\newcommand{\brho}{ \mbox{\boldmath $\rho$}}
\newcommand{\bpsi}{ \mbox{\boldmath $\psi$}}
\newcommand{\bepsilon}{ \mbox{\boldmath $\epsilon$}}
\newcommand{\bomega}{ \mbox{\boldmath $\omega$}}
\newcommand{\bDelta}{ \mbox{\boldmath $\Delta$}}
\newcommand{\bSigma}{ \mbox{\boldmath $\Sigma$}}
\newcommand{\boldeta}{ \mbox{\boldmath $\eta$}}
\newcommand{\bone}{ \mbox{\boldmath $1$}}
\newcommand{\bA}{ \mbox{\bf A}}
\newcommand{\ba}{ \mbox{\bf a}}
\newcommand{\bP}{ \mbox{\bf P}}
\newcommand{\bx}{ \mbox{\bf x}}
\newcommand{\bX}{ \mbox{\bf X}}
\newcommand{\bB}{ \mbox{\bf B}}
\newcommand{\bZ}{ \mbox{\bf Z}}
\newcommand{\by}{ \mbox{\bf y}}
\newcommand{\bY}{ \mbox{\bf Y}}
\newcommand{\bz}{ \mbox{\bf z}}
\newcommand{\bh}{ \mbox{\bf h}}
\newcommand{\br}{ \mbox{\bf r}}
\newcommand{\bt}{ \mbox{\bf t}}
\newcommand{\bs}{ \mbox{\bf s}}
\newcommand{\bb}{ \mbox{\bf b}}
\newcommand{\bL}{ \mbox{\bf L}}
\newcommand{\bu}{ \mbox{\bf u}}
\newcommand{\bg}{ \mbox{\bf g}}
\newcommand{\bv}{ \mbox{\bf v}}
\newcommand{\bV}{ \mbox{\bf V}}
\newcommand{\bW}{ \mbox{\bf W}}
\newcommand{\bG}{ \mbox{\bf G}}
\newcommand{\bH}{ \mbox{\bf H}}
\newcommand{\bD}{ \mbox{\bf D}}
\newcommand{\bK}{ \mbox{\bf K}}
\newcommand{\bM}{ \mbox{\bf M}}
\newcommand{\bw}{ \mbox{\bf w}}
\newcommand{\bo}{ \mbox{\bf o}}
\newcommand{\bfe}{ \mbox{\bf e}}
\newcommand{\alphahat}{{\hat \alpha}}
\newcommand{\iid}{\stackrel{iid}{\sim}}
\newcommand{\indep}{\stackrel{indep}{\sim}}
\newcommand{\calR}{{\cal R}}
\newcommand{\calG}{{\cal G}}
\newcommand{\calD}{{\cal D}}
\newcommand{\calS}{{\cal S}}
\newcommand{\calB}{{\cal B}}
\newcommand{\calA}{{\cal A}}
\newcommand{\calT}{{\cal T}}
\newcommand{\calO}{{\cal O}}
\newcommand{\calGP}{{\cal GP}}
\newcommand{\mR}{{\cal R}}
\newcommand{\mC}{{\cal C}}
\newcommand{\argmax}{{\mathop{\rm arg\, max}}}
\newcommand{\argmin}{{\mathop{\rm arg\, min}}}
\newcommand{\Frechet}{\mbox{Fr$\acute{\mbox{e}}$chet}}
\newcommand{\Matern}{ \mbox{Mat$\acute{\mbox{e}}$rn}}


\newcommand{\beq}{\begin{equation}}
\newcommand{\eeq}{\end{equation}}
\newcommand{\beqn}{\begin{eqnarray}}
\newcommand{\eeqn}{\end{eqnarray}}
\newtheorem{corollary}{Corollary}
\newtheorem{proposition}{Proposition}
\newtheorem{lemma}{Lemma}
\newtheorem{theorem}{Theorem}

\newtheorem{mydef}{Definition}
\newtheorem{mythm}{Theorem}
\newtheorem{mylemma}{Lemma}
\newtheorem{myproposition}{Proposition}
\newtheorem{mycor}{Corollary}


\begin{document}\linenumbers
\pagestyle{empty}
\begin{center}
{\Large {\bf Exploration and inference in spatial extremes using empirical basis functions}}\\

{\large Samuel A Morris\footnote[1]{North Carolina State University}, Brian J Reich\footnotemark[1]{}, and Emeric Thibaud\footnote[2]{Colorado State University}}

%\footnote{This is the footnote} looks like this. Later text referring to same footnote\footnotemark[\value{footnote}]
\today
\end{center}


\begin{abstract}
	words...\\
	{\bf Key words}: Bayesian data analysis; principal components analysis; max-stable process; spectral representation.

\end{abstract}
\newpage
\pagestyle{plain}
\setcounter{page}{1}

\section{Introduction}\label{ebs:intro}
The spatial Extreme Value Analysis (EVA) literature is expanding rapidly \citep{Davison2012} to meet the demands of researchers to improve estimates of rare-event probabilities by borrowing information across space and to estimate the probability of extreme events occurring simultaneously at multiple locations.
Environmental datasets commonly include observations from hundreds or thousands of locations, and advanced tools are required to explore and analyze these data.
For Gaussian data, Principle Components Analysis \citep[PCA]{Everitt2008}, also known as Empirically Orthogonal Functions \citep[EOF]{Toggweiler2001}, has proven to be a powerful tool to study correlation between spatial locations; understand the most important large-scale spatial features; and reduce the dimension of the problem to allow for simple computation even for massive datasets.
Computation and exploration are arguably more difficult for EVA than Gaussian data, yet to our knowledge no tool analogous to spatial PCA has been developed for EVA.

In EVA, extremes are separated from the bulk of the distribution by either analyzing only points above a threshold or block maximums \citep{Coles2001}, e.g., the annual maximum of the daily precipitation.
A natural spatial model for block maximum at several spatial locations is the max-stable process, which, under certain conditions, arises as the limit of the location-wise maximum of infinitely-many spatial processes \citep{deHaan2006}.
\Citet{deHaan1984} showed that any max-stable process can be represented in terms of a countable number of spatial processes (e.g., stationary log Gaussian processes), and a finite truncation of this representation has been used for conditional simulation \citep{Wang2011}.
The proposed empirical basis function (EBF) approach also uses a finite truncation of the spectral representation, and develops a method-of-moments estimator for the underlying spatial processes.

In addition to exploratory analysis, we show that the EBFs can be used for Bayesian inference on the marginal parameters at each location and to test for covariate effects.
Fully-Bayesian analysis using max-stable processes is cumbersome for large data sets \citep{Wadsworth2014,Thibaud2013a}.
One option is to use non-max-stable models that retain extremal dependence such as the skew-t process in Morris et al \hl{put on arxiv}.
Alternatively, \citet{Reich2012} propose a low-rank method based on spatial kernel functions, and others have used pairwise \citep{Padoan2010,Huser2014} and trivariate \citep{Genton2011} likelihood methods for parameter estimation.

In this paper we develop methodology to use the spectral representation of a max-stable process to identify a small set of EBFs that capture the most important spatial features of the data.
Unlike PCA/EOFs, the EBFs are not orthogonal, nonetheless these spatial functions can be plotted for exploratory analysis to reveal important spatial trends.
The EBFs can also be used in a second-stage statistical analysis.
By basing the spatial dependence on EBFs, the resulting spatial analysis does not require dubious assumptions such as stationarity.
In addition, a Bayesian analysis for either block-maximum or point above a threshold is computationally feasible for large datsets because the entire spatial process is represented by a small number of basis functions.

The paper proceeds as follows. In \sref{ebs:model} we present the low-rank model. \hl{More here about the sections}

\section{Model}\label{ebs:model}

Let $Y_{t}(\bs)$ be the observation at spatial location $\bs$ and time $t$.  We temporarily drop the subscript $t$ and describe the model for the process $Y(\bs)$ for a single time point, but return to the spatiotemporal setting in \sref{ebs:estimation}.
To focus attention on the extreme values, we emphasize the statistical model for exceedances above a location-specific threshold $T(\bs)$.
We begin by specifying a spatial model for the complete data $Y(\bs)$ and then use the censored likelihood defined by $T(\bs)$ for inference as described in \sref{ebs:MCMC}.
Although the model presented implements a censored likelihood, the model also can fit uncensored data (such as block-maxima) by setting $T(\bs) = -\infty$.

Spatial dependence is captured by modeling $Y(\bs)$ as a max-stable process \citep{deHaan2006}.
Max-stable processes have generalized extreme value (GEV; see \aref{eba:GEV}) marginal distribution.
The GEV has three parameters: location $\mu(\bs)$; scale $\sigma(\bs)$; and shape $\xi(\bs)$.
Spatial dependence is present both in the GEV parameters but also the standardized residual process
\begin{align} \label{ebeq:Y2Z}
 Z(\bs) = \left\{1+\frac{\xi(\bs)}{\sigma(\bs)}\left[Y(\bs) - \mu(\bs)\right]\right\}^{1/\xi(\bs)},
\end{align}
which has unit $\Frechet$ (i.e., GEV with location, scale, and shape all equal one) marginal distribution for all $\bs$.

Our objective is to identify a low-rank model for the spatial dependence of $Z(\bs)$.
\citep[Chapter 9]{deHaan1984} show that any max-stable process can be written as
\begin{align} \label{ebeq:spectral}
  Z(\bs) = \mbox{sup}_l B(\bs,\bt_l)A_{l}
\end{align}
where the function $B$ satisfies $B(\bs,\bt)>0$ for all $(\bs,\bt)$ and $\int B(\bs,\bt)d\bt=1$ for all $\bs$, and $(\bt_l,A_l)$ for $l=1,...,\infty$ are a Poisson process with intensity measure $dA d\bt/A^2$.

To arrive at a low-rank model, we assume there are a finite and known number of spatial basis functions $B_1(\bs), \ldots, B_L(\bs)$ that explain the important spatial variation in the process.
As in de Haan's expansion, the basis functions are restricted so that $B_l(\bs) > 0$ and $\displaystyle \sum_{l = 1}^L B_l(\bs) = 1$ for all $\bs$.
Because it is unrealistic to assume that realizations of $Z$ are exactly functions of $L$ basis functions, we include independent error variables $\epsilon(\bs)$ to capture variation not explained by the $B_l(\bs)$.
We follow \citet{Reich2012} and decompose  $Z(\bs)$ as $Z(\bs)=\theta(\bs)\varepsilon(\bs)$ where $\theta(\bs)$ is a spatial process and $\varepsilon(\bs)\iid$ GEV$(1,\alpha,\alpha)$ is independent error.
The spatial component is
\begin{align} \label{ebeq:theta}
  \theta(\bs) = \left(\sum_{l=1}^LB_{l}(\bs)^{1/\alpha}A_{l}\right)^{\alpha}.
\end{align}
If $B_{l}(\bs)>0$, $\sum_{l=1}^LB_{l}(\bs)=1$ for all $\bs$, and the $A_{l}$ have positive stable (PS; \aref{eba:gridapprox}) distribution $A_{l}\iid$ PS$(\alpha)$, then $Z(\bs)$ is max-stable and has unit $\Frechet$ marginal distributions.

Extremal spatial dependence for max-stable processes can be summarized by the extremal coefficient \citep[EC]{Schlather2003} $\vartheta(\bs,\bt)\in[1,2]$, where
\begin{align} \label{ebeq:ECdev}
  \mbox{Prob}[Z(\bs)<c,Z(\bt)<c] = \mbox{Prob}[Z(\bs)<c]^{\vartheta(\bs,\bt)}.
\end{align}
For the PS random effects model the EC has the form
\begin{align} \label{ebeq:EC}
   \vartheta(\bs,\bt) = \sum_{l=1}^L \left[B_{l}(\bs)^{1/\alpha}+B_{l}(\bs)^{1/\alpha}\right]^\alpha.
\end{align}
In particular, $\vartheta(\bs,\bs) = 2^{\alpha}$ for all $\bs$.

\section{Estimating the spatial dependence function}\label{ebs:estimation}
To estimate the extremal coefficient function, we consider the process at $n_s$ spatial locations $\bs_1,...,\bs_{n_s}$ and $n_t$ times $t=1,...,n_t$.
The basis functions are fixed over time, but the random effects and errors are independent over time.
That is $Z_t(\bs) = \theta_t(\bs) \epsilon_t(\bs)$ where $\theta_t(\bs) = \left(\displaystyle \sum_{l=1}^L B_{l}(\bs)^{1/\alpha}A_{lt}\right)^{\alpha}$, $A_{lt} \iid$ PS$(\alpha)$, and $\epsilon_t(\bs) \iid$ GEV$(1, \alpha, \alpha)$.
Denote $Y_t(\bs_i) = Y_{it}$, $B_l(\bs_i) = B_{il}$, $T(\bs_i)=T_i$, and $\vartheta(\bs_i,\bs_j) = \vartheta_{ij}$.

In this section we develop an algorithm to estimate the spatial dependence parameter $\alpha$ and the $n_s\times L$ matrix $\bB = \{B_{il}\}$.
% Given these parameters, we insert them into our model and proceed with Bayesian analysis as described in \sref{ebs:MCMC}.
Our algorithm has the following steps:
\begin{enumerate}[(1)]
  \item Obtain an initial estimate of the extremal coefficient for each pair of locations, ${\hat \vartheta}_{ij}$.
  \item Spatially smooth these initial estimates ${\hat \vartheta}_{ij}$ using kernel smoothing to obtain ${\tilde \vartheta}_{ij}$.
  \item Estimate the spatial dependence parameters by minimizing the difference between model-based coefficients, $\vartheta_{ij}$, and smoothed coefficients, ${\tilde \vartheta}_{ij}$.
\end{enumerate}

% Use for thesis
The first-stage estimates are obtained using an empirical estimate as follows.
To estimate the spatial dependence we first remove variation in the marginal distribution.
Let $U_{it} = \sum_{k=1}^{n_t} I[Y_{ik}<Y_{it}]/n_t$, so that the $U_{it}$ are approximately uniform at each location.
Then for some extreme probability $q\in(0,1)$, solving \eref{ebeq:ECdev} suggests the estimate
\begin{align}\label{ebeq:EChat0}
   {\hat \vartheta}_{ij}(q) = \frac{\log[Q_{ij}(q)]}{\log(q)},
\end{align}
where $Q_{ij}(q) = \sum_{t=1}^{n_t}I[U_{it}<q,U_{jt}<q]/n_t$ is the sample proportion of the time points at which both sites are less than $q$.
Since all large $q$ give valid estimates, we average over a grid of $q$ with $q_1<...<q_{n_q}$
\begin{align} \label{ebeq:EChat1}
{\hat \vartheta}_{ij} = \frac{1}{n_q}\sum_{j=1}^{n_q}{\hat \vartheta}_{ij}(q_j).
\end{align}

% Use for paper
% The first-stage estimates are obtained from the estimator of \citet{Schlather2003} using the \texttt{fitextcoeff} function in the \texttt{SpatialExtremes} \citep{Ribatet2015} package of \texttt{R} \citep{Rmanual}.
Assuming the true EC is smooth over space, the initial estimates ${\hat \vartheta}_{ij}$ can be improved by smoothing.
Let
\begin{align} \label{ebeq:EChat2}
  {\tilde \vartheta}_{ij} = \frac{\sum_{u=1}^{n_s}\sum_{v=1}^{n_s} w_{iu}w_{jv}{\hat \vartheta}_{uv}}
  {\sum_{u=1}^{n_s}\sum_{v=1}^{n_s} w_{iu}w_{jv}},
\end{align}
where $w_{iu} = \exp[-(||\bs_i-\bs_u'||/\phi)^2])$ is the Gaussian kernel function with bandwidth $\phi$.
The elements ${\hat \vartheta}_{ii}$ do not contribute any information as ${\hat \vartheta}_{ii}=1$ for all $i$ by construction.
To eliminate the influence of these estimates we set $w_{ii}=0$.
However, this approach does give imputed values ${\tilde \vartheta}_{ii}$, which provide information about small-scale spatial variability.

The dependence parameters $B_{lt}$ and $\alpha$ are estimated by comparing estimates ${\tilde \vartheta}_{ij}$ with the model-based values $\vartheta_{ij}$.
For all $i$, $\vartheta_{ii} = 2^{\alpha}$, and therefore we set $\alpha$ to $\alphahat = \log_2(\sum_{i=1}^{n_s}{\tilde \vartheta}_{ii}/n_s)$.
Given $\alpha=\alphahat$, it remains to estimate $\bB$.
Similarly to \citet{Smith1990} for a stationary max-stable process, we use squared-error loss, so the estimate $\hat{\bB}$ is the minimizer of
\begin{align} \label{ebeq:Bhat}
\sum_{i<j} \left({\tilde \vartheta}_{ij} - \vartheta_{ij}\right)^2
  =
  \sum_{i<j} \left({\tilde \vartheta}_{ji} - \sum_{l=1}^L[B_{il}^{1/\alphahat} + B_{jl}^{1/\alphahat}]^{\alphahat}\right)^2
\end{align}
under the restrictions that $B_{il}\ge 0$ for all $i$ and $l$ and $\sum_{l=1}^LB_{il}=1$ for all $i$.
Since the minimizer of \eref{ebeq:Bhat} does not have a closed form, we use block coordinate descent to obtain ${\hat \bB}$.
We cycle through spatial locations and update the vectors $(B_{i1},...,B_{iL})$ conditioned on the values for the other location and repeat until convergence.
At each step, we use the restricted optimization routine in the \texttt{R} function \texttt{optim}.
This algorithm gives estimates of the $B_{il}$ at the $n_s$ data locations, but is easily extended to all $\bs$ for spatial prediction.
The kernel smoothing step ensures that the estimates for $\hat{B}_{il}$ are spatially smooth, and thus interpolation of the $\hat{B}_{il}$ gives spatial functions $\hat{B}_l(\bs)$.

These functions provide useful exploratory data analysis techniques.
Maps of $\hat{B}_l(\bs)$ show important spatial features in the extremal dependence.
Furthermore, they allow for a non-stationary spatial dependence structure.
The relative contribution of each term can be measured by
\begin{align} \label{ebeq:v}
v_l = \frac{1}{n_s}\sum_{i=1}^{n_s}{\hat B}_{il}.
\end{align}
Since $\sum_{l=1}^L{\hat B}_{il}=1$ for all $i$, we have $\sum_{l=1}^Lv_l = 1$.
Therefore, terms with large $v_l$ are the most important.
The order of the terms is arbitrary, and so we reorder the terms so that $v_1\ge \ldots \ge v_L$.

\section{Bayesian implementation details}\label{ebs:MCMC}
For our data analysis in \sref{ebs:analysis} we allow the GEV location and scale parameters, denoted $\mu_t(\bs)$ and scale $\sigma_t(\bs)$ respectively, to vary with space and time.
The model we choose is as follows
\begin{align}\label{ebeq:GPprior}
  \mu_t(\bs) &= \beta_{1, \text{int}}(\bs) + \beta_{1, \text{time}}(\bs) t\\
  \log[\sigma_t(\bs)] &= \beta_{2, \text{int}}(\bs) + \beta_{2, \text{time}}(\bs) t
\end{align}
where
\begin{align} \label{ebeq:GPhyper}
  \beta_{1, \text{int}}(\bs) &\sim \text{N}(\mu^{\phantom{2}}_{1, \text{int}}, \sigma^2_{1, \text{int}} \bSigma) \quad &\beta_{1, \text{time}}(\bs) &\sim \text{N}(\mu^{\phantom{2}}_{1, \text{time}}, \sigma^2_{1, \text{time}} \bSigma) \\
  \beta_{2, \text{int}}(\bs) &\sim \text{N}(\mu^{\phantom{2}}_{2, \text{int}}, \sigma^2_{2, \text{int}} \bSigma) \quad &\beta_{2, \text{time}}(\bs) &\sim \text{N}(\mu^{\phantom{2}}_{2, \text{time}}, \sigma^2_{2, \text{time}} \bSigma) \nonumber
\end{align}
are Gaussian process priors and $\bSigma$ is an exponential spatial correlation matrix obtained from \mbox{$\rho(h) = \exp\left\{- \frac{h}{\phi}\right\}$} where $h = ||\bs_1 - \bs_2||$ is the Euclidean distance between sites $\bs_1$ and $\bs_2$.
The GEV shape parameter $\xi$ is held constant over space and time because this parameter is challenging to estimate.
Collectively, let the marginal GEV parameters at location $i$ and time $t$ be $\Theta_{it} = \{\mu_{it},\sigma_{it},\xi\}$ where $\mu_{it} = \mu_t(\bs_i)$ and $\sigma_{it} = \sigma_t(\bs_i)$.

As shown in \citet{Reich2012}, the uncensored responses $Y_{t}(\bs)$ are conditionally independent given the spatial random effects, with conditional distribution
\begin{align} \label{ebeq:Ycond}
   Y_{it}|\theta_{it},\Theta_{it}\indep GEV(\mu^*_{it}, \sigma_{it}^*,\xi^*),
\end{align}
where $\mu_{it}^* = \mu_{it} + \frac{\sigma_{it}}{\xi}(\theta_{it}^\xi - 1)$,
$\sigma_{it}^* = \alpha\sigma_{it}\theta_{it}^\xi$, and $\xi^* = \alpha\xi$.
Therefore, the conditional likelihood conveniently factors across observations; marginalizing over the random effect $\theta_{it}$ induces extremal spatial dependence.
To focus on the extreme values above the local threshold $T_i$, we use the censored likelihood
\begin{align} \label{ebeq:g}
d(y;\theta_{it},\Theta_{it}, T_i)  =
\left\{\begin{array}{ll}
    F(y;\mu_{it}^*,\sigma_{it}^*,\xi^*) & y \le T_i \\
  f(y;\mu_{it}^*,\sigma_{it}^*,\xi^*) & y>T_i,
\end{array}\right.
\end{align}
where $F$ and $f$ are the GEV distribution and density functions, respectively, defined in \aref{eba:evd}.

In summary, given the estimates of $\alpha$ and $\bB$, the hierarchical model is
\begin{align} \label{ebeq:bayesmodel}
  Y_{it} |\theta_{ij} &\indep d(y;\theta_{it},\Theta_{it}, T_i) \\
  \theta_{it} &= \left(\sum_{l=1}^L{\hat B}_{il}^{1/\alphahat}A_{lt}\right)^{\alphahat}
  \text{\ \ \ where \ \ \ }
  A_{lt} \iid PS(\alphahat)\nonumber\\
  \mu_{it} &= \beta_{1, \text{int}}(\bs_i) + \beta_{1, \text{time}}(\bs_i) t \nonumber \\
  \log(\sigma_{it}) &= \beta_{2, \text{int}}(\bs) + \beta_{2, \text{time}}(\bs) t. \nonumber
\end{align}
We estimate parameters $\Theta = \left\{A_{lt}, \bbeta_1, \bbeta_2, \xi \right\}$ using Markov chain Monte Carlo methods.
We use a Metropolis-Hastings algorithm to update the model parameters with random walk candidate distributions for all parameters.
The PS density is challenging to evaluate as it does not have a closed form.
One technique to avoid this complication is to incorporate auxiliary random variables \citep{Stephenson2009}, but we opt for a numerical approximation to the integral as described in \aref{eba:gridapprox}.
The hyperparameters $\mu_{1, \text{int}}, \mu_{1, \text{time}}, \mu_{2, \text{int}}, \mu_{2, \text{time}}$ and $\sigma^2_{1, \text{int}}, \sigma^2_{1, \text{time}}, \sigma^2_{2, \text{int}}, \sigma^2_{2, \text{time}}$ are updated using Gibbs sampling since their prior distributions are conjugate.

The first-stage estimate of the extremal coefficients has three tuning parameters: the quantile thresholds $q_1,...,q_{n_q}$, the kernel bandwidth $\phi$, and the number of terms $L$.
In \sref{ebs:analysis} we explore a few possibilities for $L$ and discuss sensitivity to this choice.
The second-stage Bayesian analysis requires selecting thresholds $T_i,...,T_{n_s}$.  For this we use spatially smoothed sample quantiles.
That is, we set $T_i$ to the 0.95 quantile of the $Y_{it}$ and its five nearest neighbors.

\section{Data analysis}\label{ebs:analysis}
In this section, we illustrate our method with two data analyses.
In \sref{ebs:georgia}, we present a points above a threshold analysis using annual acreage burned due to wildfires in Georgia from 1965 -- 2014.
This is followed in \sref{ebs:precip} by an analysis of block maxima precipitation data in the eastern U.S.
We compare our method with another method that uses standardized Gaussian kernels for the spatial basis functions.

\subsection{Gaussian kernel basis functions}
To provide a comparison of our model with another approach, we also fit a model that uses standardized Gaussian kernels for the spatial basis functions \citep{Reich2012}.
In this method, \citeauthor{Reich2012} introduce a set of $\bk_1, \ldots, \bk_L$ spatial knots and use standardized Gaussian kernel functions (GSK) instead of using EBFs for the $\hat{B_l}(\bs)$.
For the comparison between EBF and GSK methods, we use the same number of basis functions.
We obtain estimates the spatial bandwidth $\hat{\rho}$ and spatial dependence $\hat{\alpha}$, using the same block coordinate descent as with the EBF method, and treat these as fixed in the MCMC.
For more information about the basis functions, see \aref{eba:gskfunctions}.

\subsection{Analysis of extreme Georgia fires}\label{ebs:georgia}
The dataset used for our application is composed of yearly acreage burned due to wildfires for each county in Georgia from 1965 -- 2014 (\texttt{http://weather.gfc.state.ga.us/FireData/}).
\fref{ebfig:firets25} shows the time series of $\log$(acres burned) for 25 randomly selected counties.
Based on this plot and other exploratory analysis, we see no evidence of non-linear trends and proceed with linear time trends for the GEV location and scale parameters.

\begin{figure}[htbp]  % markdown/eda/eda-plotting.R
  \centering
  \includegraphics[width=0.55\linewidth]{plots/fire-spag-rand-25}
  \includegraphics[width=0.44\linewidth, trim = 0 8em 0 10em]{plots/fire-spatial-q95.pdf}
  \caption{Time series of log acres burned for 25 randomly selected counties with colors coding the county's quadrant (left), and spatially smoothed threshold values, $T_i$ for each county (right).}
  \label{ebfig:firets25}
\end{figure}

We estimate the extremal coefficient function $\hat{\theta}_{ij}$ by setting $q_1 = 0.90$ and using $n_q = 100$.
With more data, it would possible to increase $q_1$, but we set $q_1 = 0.90$ to increase the stability when estimating $\hat{\vartheta}_{ij}$.
Because these data are not block-maxima, we select a site-specific threshold $T_i$ to use in the analysis with the following algorithm.
Without some adjustment to the data, it is challenging to borrow information across sites to inform the threshold selection.
We first standardize the data, separated by county, by subtracting the site's median and dividing by the site's interquartile range.
Denote the standardized data by $\widetilde{\bY}_i$.
% \begin{align}
%   \widetilde{\bY}_i = \frac{\bY_i - \text{med}(\bY_i)}{\text{IQR}(\bY_i)}
% \end{align}
% where med$(\cdot)$ is the median, and IQR$(\cdot)$ is the inter-quartile range.
Then we combine all sites together and plot a mean residual plot for $\widetilde{Y}_{it}, i = 1, \ldots, n_s$ and $t = 1, \ldots, n_t$.
The mean residual plot is given in \fref{ebfig:mrlthresh}.
Based upon the mean residual plot, we select the 95th percentile for the threshold.
To calculate $T_i$ for each county, we use the 95th percentile for the combined data for county $i$ and its five closest counties (see \fref{ebfig:firets25}).

\begin{figure}[htbp]  % markdown/eda/eda-plotting.R
  \centering
  \includegraphics[width = \linewidth]{plots/fire-mrl-plots.pdf}
  \caption{Mean residual plot for the data pooled across counties after standardizing using the county's median and interquartile range. The two panels show different ranges on the x-axis and include a vertical line at the sample 95th percentile.}
  \label{ebfig:mrlthresh}
\end{figure}

%\begin{figure}[htbp]
%  \centering
%  \includegraphics[width = 0.47\linewidth, trim = 0 10em 0 10em]{plots/fire-spatial-q95.pdf}
%  \caption{Spatially smoothed threshold values for each county.}
%  \label{fig:mrlthresh}
%\end{figure}

The empirical basis functions for the analysis can be used to help explore spatial dependence in the extremes.
See \fref{ebfig:fireebfs} for the first six (of 25) EBFs for the Georgia fire data.
\begin{figure}[htbp] % markdown/fire-analysis/basis-functions.R
  \centering
  \includegraphics[width=\linewidth]{plots/fire-ebf-panel.pdf}
  \caption{First six EBFs for the Georgia fire data.}
  \label{ebfig:fire-ebfpanel}
\end{figure}
We also plot the cumulative sum of the contributions for each basis function in \ref{ebfig:fire-v25}.
\begin{figure}[htbp] % markdown/fire-analysis/basis-functions.R
  \centering
  \includegraphics[width=0.5\linewidth]{plots/firev-25.pdf}
  \caption{Cumulative sum of contributions $v_1, \ldots, v_25$.}
  \label{ebfig:fire-v25}
\end{figure}
As a comparison, we provide the first six principal components of the fire data along with the cumulative sum of the first 25 eigenvalues in \fref{ebfig:fire-eigpanel}.


\subsection{Results for fire analysis}\label{ebs:results-fire}
We use 10-fold cross-validation to assess the predictive performance of a model.
For each method, we randomly select 90\% of the observations across counties and years to be used as a training set to fit the model.
The remaining 10\% of sites and years are withheld for testing model predictions.
To assess the predictions for the test set, we use quantile scores and Brier scores \citep{Gneiting2007}.
The quantile score is given by \hl{give formula}.
The Brier score is given by \hl{give formula}.
For both of these methods, a lower score indicates a better fit.
The Brier and quantile scores for the fire analysis are given in \tref{ebtbl:firescores}.

% Posterior summaries for each county's $\beta_\text{time}$ coefficient are given in \fref{ebfig:ebfpost} and \fref{ebfig:gskpost}.
% These plots both seem to catch similar features with some differences particularly in the posterior distribution of the county-specific $\beta_{\mu, \text{time}}$.

\begin{table}[htbp]
\caption{Average Brier scores ($\times 100$) for selected thresholds, quantile scores for selected quantiles, and timing for fire analysis.}
\label{ebtbl:firescores}
\centering
  \begin{tabular}{llcrrcrrcc}
  \toprule
  & & \phantom{ab} & \multicolumn{2}{c}{Brier Scores ($\times 100$)} & \phantom{abc} & \multicolumn{2}{c}{Quantile Scores} & \phantom{ab} & \\
  \cmidrule{4-5} \cmidrule{7-8}
  & Process && $q(0.95)$ & $q(0.99)$ && $q(0.95)$ & $q(0.99)$ && Time (in hours)\\
  \midrule
  L = 5  & EBF &&  &  &&  &  && \\
         & GSK &&  &  &&  &  && \\
  \midrule
  L = 10 & EBF &&  &  &&  &  && \\
         & GSK &&  &  &&  &  && \\
  \midrule
  L = 15 & EBF &&  &  &&  &  && \\
         & GSK &&  &  &&  &  && \\
  \midrule
  L = 20 & EBF &&  &  &&  &  && \\
         & GSK &&  &  &&  &  && \\
  \midrule
  L = 25 & EBF &&  &  &&  &  && \\
         & GSK &&  &  &&  &  && \\
  \midrule
  L = 30 & EBF &&  &  &&  &  && \\
         & GSK &&  &  &&  &  && \\
  \midrule
  L = 35 & EBF &&  &  &&  &  && \\
         & GSK &&  &  &&  &  && \\
  \midrule
  L = 40 & EBF &&  &  &&  &  && \\
         & GSK &&  &  &&  &  && \\
  \bottomrule
	\end{tabular}
\end{table}

\hl{Figure here with BS and QS for the fire data}

Based on the cross-validation results, we run a full analysis using all of the data with $L = 25$.
\fref{ebfig:fire-ebfpostbeta1} and \fref{ebfig:fire-gskpostbeta1} give the county by county posterior means of $\beta_{1, \text{time}}$ and $\beta_{2, \text{time}}$ for EBF and GSK respectively.
\fref{ebfig:fire-ebfpostbeta1pos} and \fref{ebfig:fire-gskpostbeta1pos} give the county by county posterior P$(\beta_{1, \text{time}} > 0)$ and P$(\beta_{2, \text{time}} > 0)$ for EBF and GSK respectively.

\begin{figure}[htbp]  % markdown/fire-analysis/posterior_map.R
  \centering
  \includegraphics[width=\linewidth]{plots/fire-ebf-post-betatime.pdf}
  \caption{Posterior mean of $\beta_{1, \text{time}}$ (left) and $\beta_{2, \text{time}}$ (right) for fire data using EBF.}
  \label{ebfig:fire-ebfpostbeta1}
\end{figure}

\begin{figure}[htbp]  % markdown/fire-analysis/posterior_map.R
  \centering
  \includegraphics[width=\linewidth]{plots/fire-gsk-post-betatime.pdf}
  \caption{Posterior mean of $\beta_{1, \text{time}}$ (left) and $\beta_{2, \text{time}}$ (right) for fire data using GSK.}
  \label{ebfig:fire-gskpostbeta1}
\end{figure}

\begin{figure}[htbp]  % markdown/fire-analysis/posterior_map.R
  \centering
  \includegraphics[width=\linewidth]{plots/fire-ebf-post-betatimepos.pdf}
  \caption{Posterior P$(\beta_{1, \text{time}} > 0)$ (left) and P$(\beta_{2, \text{time}} > 0)$ (right) for fire data using EBF.}
  \label{ebfig:fire-ebfpostbeta1pos}
\end{figure}

\begin{figure}[htbp]  % markdown/fire-analysis/posterior_map.R
  \centering
  \includegraphics[width=\linewidth]{plots/fire-gsk-post-betatimepos.pdf}
  \caption{Posterior P$(\beta_{1, \text{time}} > 0)$ (left) and P$(\beta_{2, \text{time}} > 0)$ (right) for fire data using GSK.}
  \label{ebfig:fire-gskpostbeta1pos}
\end{figure}

\hl{Is this what you were thinking for the difference in $q(0.90)$? These do not use $\theta$.}
We also plot the difference in $\hat{q}(0.90)$, the estimated 90th quantile, between 2014 and 1965.
To obtain the estimate for each county, we first take the posterior mean of $\beta_{1, \text{int}}$, $\beta_{1, \text{time}}$, $\beta_{2, \text{int}}$, and $\beta_{2, \text{time}}$ for each county.
Then we find $\mu(\bs, t)$ and $\sigma(\bs, t)$ for each county in 1965 and 2014.
Then for the shape parameter we take the posterior mean of $\xi$.
Finally, we use the inverse distribution function of the GEV to obtain $\hat{q}(0.90)_{1965}$ and $\hat{q}(0.90)_{2014}$ and plot \mbox{$\hat{q}(0.90)_\text{diff} = \hat{q}(0.90)_{2014} - \hat{q}(0.90)_{1965}$} in \fref{ebfig:fire-q90diff}.

\begin{figure}[htbp]  % markdown/fire-analysis/posterior_map.R
  \centering
  \includegraphics[width=\linewidth]{plots/fire-q90diff-compare.pdf}
  \caption{Difference in $q(0.90)$ for fire data between 2014 and 1965 for EBF (left) and GSK (right). Three counties (labeled) have differences higher than 2500. County 1: Ware (EBF: 9690, GSK: 8934), County 2: Clinch (EBF: 6693, GSK: 6520), and County 3: Charlton (EBF: 6084, GSK: 6794)}
  \label{ebfig:fire-q90diff}
\end{figure}

% \begin{figure}  % markdown/fire-analysis/combine-tables.R
%   \centering
%   \includegraphics[width=\linewidth]{plots/ebf-post-betatime.pdf}
%   \caption{Posterior summaries of $\beta_{\text{time}}$ when using EBF for the spatial process with $L = 35$.}
%   \label{ebfig:ebfpost}
% \end{figure}

% \begin{figure}  % markdown/fire-analysis/combine-tables.R
%   \centering
%   \includegraphics[width=\linewidth]{plots/gsk-post-betatime.pdf}
%   \caption{Posterior summaries of $\beta_{\text{time}}$ when using GSK for the spatial process with $L = 35$.}
%   \label{ebfig:gskpost}
% \end{figure}

% \begin{figure}  % markdown/fire-analysis/combine-tables.R
%   \centering
%   \includegraphics[width=0.47\linewidth]{plots/fire-bs}
%   \includegraphics[width=0.47\linewidth]{plots/fire-qs}
%   \caption{Average Brier score for exceeding q(0.95) and q(0.99) (left). Average Quantile score for exceeding q(0.95) and q(0.99) (right).}
%   \label{fig:avgqscore}
% \end{figure}

% \begin{figure}  % markdown/fire-analysis/combine-tables.R
% 	\centering
% 	\includegraphics[width=\linewidth]{plots/fire-qs}
% 	\caption{Average quantile score for q(0.95) (left). Average quantile score for q(0.99) (right).}
%   \label{fig:avgqscore}
% \end{figure}

% \begin{figure}  % markdown/fire-analysis/combine-tables.R
% 	\centering
% 	\includegraphics[width=0.47\linewidth]{plots/fire-timing}
% 	\caption{Timing comparison of basis functions to kernel functions for the spatial process (100 iterations)}
%   \label{fig:timingcompare}
% \end{figure}

\subsection{Analysis of annual precipitation}\label{ebs:precip}
We also conduct an analysis of the precipitation data presented in \citep{Reich2012}.
The data are climate model output from the North American Regional Climate Change Assessment Program (NARCCAP).
This data consists of $n_s = 697$ grid cells at a 50km resolution in the eastern US, and includes historical data (1969 -- 2000) as well as future conditions (2039 -- 2070).

\begin{figure}[htbp]  % markdown/precipitation/cv-setup.R
  \centering
  \includegraphics[width=\linewidth]{plots/precip-ts}
  \caption{Time series of yearly max precipitation for current (1969 -- 2000) (left). Time series of yearly max precipitation for future (2039 -- 2070) (right).}
  \label{ebf:tsprecip}
\end{figure}

For this dataset, to estimate the EBFs, we use the combined current and future data.
The first six EBFs for the combined data are given in \fref{ebfig:precip-ebfpanel}.
\begin{figure}[htbp]  % markdown/precipitation/cv-setup.R
  \centering
  \includegraphics[width=\linewidth]{plots/precip-ebf-panel.pdf}
  \caption{First six EBFs for the combined precipitation data.}
  \label{ebfig:precip-ebfpanel}
\end{figure}
We also plot the cumulative sum of the contributions for each basis function in \ref{ebfig:precip-v25}.
\begin{figure}[htbp] % markdown/fire-analysis/basis-functions.R
  \centering
  \includegraphics[width=0.5\linewidth]{plots/precipv-25.pdf}
  \caption{Cumulative sum of contributions $v_1, \ldots, v_25$.}
  \label{ebfig:precip-v25}
\end{figure}
As a comparison, we provide the first six principal components of the fire data along with the cumulative sum of the first 25 eigenvalues in \fref{ebfig:fire-eigpanel}.

\subsection{Results for precipitation analysis}\label{ebs:results-precip}

\begin{table}[htbp]
\caption{Average Brier scores ($\times 100$) for selected thresholds, quantile scores for selected quantiles, and timing for current precipitation analysis.}
\label{ebtbl:firescores}
\centering
  \begin{tabular}{llcrrcrrcc}
  \toprule
  & & \phantom{ab} & \multicolumn{2}{c}{Brier Scores ($\times 100$)} & \phantom{abc} & \multicolumn{2}{c}{Quantile Scores} & \phantom{ab} & \\
  \cmidrule{4-5} \cmidrule{7-8}
  & Process && $q(0.95)$ & $q(0.99)$ && $q(0.95)$ & $q(0.99)$ && Time (in hours)\\
  \midrule
  L = 5  & EBF &&  &  &&  &  && \\
         & GSK &&  &  &&  &  && \\
  \midrule
  L = 10 & EBF &&  &  &&  &  && \\
         & GSK &&  &  &&  &  && \\
  \midrule
  L = 15 & EBF &&  &  &&  &  && \\
         & GSK &&  &  &&  &  && \\
  \midrule
  L = 20 & EBF &&  &  &&  &  && \\
         & GSK &&  &  &&  &  && \\
  \midrule
  L = 25 & EBF &&  &  &&  &  && \\
         & GSK &&  &  &&  &  && \\
  \midrule
  L = 30 & EBF &&  &  &&  &  && \\
         & GSK &&  &  &&  &  && \\
  \midrule
  L = 35 & EBF &&  &  &&  &  && \\
         & GSK &&  &  &&  &  && \\
  \midrule
  L = 40 & EBF &&  &  &&  &  && \\
         & GSK &&  &  &&  &  && \\
  \bottomrule
  \end{tabular}
\end{table}

\begin{table}[htbp]
\caption{Average Brier scores ($\times 100$) for selected thresholds, quantile scores for selected quantiles, and timing for future precipitation analysis.}
\label{ebtbl:firescores}
\centering
  \begin{tabular}{llcrrcrrcc}
  \toprule
  & & \phantom{ab} & \multicolumn{2}{c}{Brier Scores ($\times 100$)} & \phantom{abc} & \multicolumn{2}{c}{Quantile Scores} & \phantom{ab} & \\
  \cmidrule{4-5} \cmidrule{7-8}
  & Process && $q(0.95)$ & $q(0.99)$ && $q(0.95)$ & $q(0.99)$ && Time (in hours)\\
  \midrule
  L = 5  & EBF &&  &  &&  &  && \\
         & GSK &&  &  &&  &  && \\
  \midrule
  L = 10 & EBF &&  &  &&  &  && \\
         & GSK &&  &  &&  &  && \\
  \midrule
  L = 15 & EBF &&  &  &&  &  && \\
         & GSK &&  &  &&  &  && \\
  \midrule
  L = 20 & EBF &&  &  &&  &  && \\
         & GSK &&  &  &&  &  && \\
  \midrule
  L = 25 & EBF &&  &  &&  &  && \\
         & GSK &&  &  &&  &  && \\
  \midrule
  L = 30 & EBF &&  &  &&  &  && \\
         & GSK &&  &  &&  &  && \\
  \midrule
  L = 35 & EBF &&  &  &&  &  && \\
         & GSK &&  &  &&  &  && \\
  \midrule
  L = 40 & EBF &&  &  &&  &  && \\
         & GSK &&  &  &&  &  && \\
  \bottomrule
  \end{tabular}
\end{table}

\hl{Figure here with BS and QS for the precipitation data}

Based on the cross-validation results, we run a full analysis using all of the data with $L = 25$.
\fref{ebfig:fire-ebfpostbeta1} and \fref{ebfig:fire-gskpostbeta1} give the county by county posterior means of $\beta_{1, \text{time}}$ and $\beta_{2, \text{time}}$ for EBF and GSK respectively.
\fref{ebfig:fire-ebfpostbeta1pos} and \fref{ebfig:fire-gskpostbeta1pos} give the county by county posterior P$(\beta_{1, \text{time}} > 0)$ and P$(\beta_{2, \text{time}} > 0)$ for EBF and GSK respectively.

\begin{figure}[htbp]  % markdown/precipitation/posterior_map.R
  \centering
  \includegraphics[width=\linewidth]{plots/precip-ebf-post-betatime.pdf}
  \caption{Posterior mean of $\beta_{1, \text{time}}$ (left) and $\beta_{2, \text{time}}$ (right) for current (top) and future (bottom) precipitation data using EBF.}
  \label{ebfig:fire-ebfpostbeta1}
\end{figure}\tabularnewline

\begin{figure}[htbp]  % markdown/precipitation/posterior_map.R
  \centering
  \includegraphics[width=\linewidth]{plots/precip-gsk-post-betatime.pdf}
  \caption{Posterior mean of $\beta_{1, \text{time}}$ (left) and $\beta_{2, \text{time}}$ (right) for current (top) and future (bottom) precipitation data using GSK.}
  \label{ebfig:fire-gskpostbeta1}
\end{figure}

\begin{figure}[htbp]  % markdown/precipitation/posterior_map.R
  \centering
  \includegraphics[width=\linewidth]{plots/precip-ebf-post-betatimepos.pdf}
  \caption{Posterior P$(\beta_{1, \text{time}} > 0)$ (left) and P$(\beta_{2, \text{time}} > 0)$ (right) for current (top) and future (bottom) precipitation data using EBF.}
  \label{ebfig:fire-ebfpostbeta1pos}
\end{figure}

\begin{figure}[htbp]  % markdown/precipitation/posterior_map.R
  \centering
  \includegraphics[width=\linewidth]{plots/precip-gsk-post-betatimepos.pdf}
  \caption{Posterior P$(\beta_{1, \text{time}} > 0)$ (left) and P$(\beta_{2, \text{time}} > 0)$ (right) for current (top) and future (bottom) precipitation data using GSK.}
  \label{ebfig:fire-gskpostbeta1pos}
\end{figure}

We also plot the difference in $\hat{q}(0.90)$, the estimated 90th quantile, between 2000 and 1969 for current and 2070 and 2039 for future.
The estimates are obtained in the same was as for the fire analysis and are given in \fref{ebfig:fire-q90diff}.

\begin{figure}[htbp]  % markdown/precipitation/posterior_map.R
  \centering
  \includegraphics[width=\linewidth]{plots/precip-q90diff-compare.pdf}
  \caption{Difference in $q(0.90)$ for precipitation data between 2000 and 1969 for current (top) and 2070 and 2039 for future (bottom) using EBF (left) and GSK (right).}
  \label{ebfig:precip-q90diff}
\end{figure}

% \begin{figure}  % markdown/precipitation/combine-tables.R
%   \centering
%   \includegraphics[width=\linewidth]{plots/precip-bs.pdf}
%   \caption{Brier scores for current and future precipitation analysis.}
%   \label{ebfig:precip-bs}
% \end{figure}

% \begin{figure}  % markdown/precipitation/combine-tables.R
%   \centering
%   \includegraphics[width=\linewidth]{plots/precip-qs.pdf}
%   \caption{Quantile scores for current and future precipitation analysis.}
%   \label{ebfig:precip-qs}
% \end{figure}

% \begin{figure}  % markdown/precipitation/combine-tables.R
%   \centering
%   \includegraphics[width=\linewidth]{plots/precip-post-time.pdf}
%   \caption{Posterior distributions for $\beta_{\text{time}}$ for $\mu$ (left) and $\log(\sigma)$ (right).}
%   \label{ebfig:precip-qs}
% \end{figure}

\section{Conclusions}\label{ebs:con}

\section*{Acknowledgements}
The authors would like to acknowledge Dan Cooley for his helpful suggestions on the manuscript.

\appendix
\section{Appendices}
\subsection{Extreme value distributions} \label{eba:GEV}
The cumulative distribution function for the GEV is $F(y) = \exp\{-t(y)\}$ where
\begin{align} \label{ebeq:gevt}
  t(y) = \left\{ \begin{array}{ll}
      \left[1 + \xi \displaystyle \frac{y - \mu}{\sigma}\right]^{-1 / \xi}, \quad &\xi \neq 0 \\ [0.5em]
      \exp\left\{- \displaystyle \frac{y - \mu}{\sigma}\right\}, &\xi = 0.
  \end{array}\right.
\end{align}
The probability density function for the GEV is given by $f(y) = \frac{1}{\sigma} t(x)^{\xi + 1} \exp\{-t(y)\}$ where $t(y)$ is defined in \eref{ebeq:gevt}.

\subsection{Grid approximation to PS density} \label{eba:gridapprox}
The PS($\alpha$) density can be challenging to use because it does not have a closed form.
From Section 2 of \citep{Stephenson2009}, the density can be expressed as
\begin{align}
  g_1(A) = \int_0^1 g_1(A, B) dB,
\end{align}
where
\begin{align}
  g_1(A, B) = \frac{\alpha}{1 - \alpha} \left( \frac{1}{A} \right)^{1 / 1 - \alpha} c(\pi B) \exp \left\{ -\left(\frac{1}{A}\right)^{\alpha / (1 - \alpha)} c(\pi B) \right\},
\end{align}
with
\begin{align}
  c(\psi) = \left[\frac{\sin(\alpha \psi)}{\sin(\psi)}\right]^{1 / (1 - \alpha)} \frac{\sin[(1 - \alpha) \psi]}{\sin(\alpha \psi)}.
\end{align}
\Citet{Stephenson2009} presents an auxiliary variable technique to deal with the integral in the density function, but we opt to numerically evaluate the integral because it is only one-dimensional.
To evaluate the integral, we use 50 evenly spaced quantiles of a Beta$(0.5, 0.5)$ distribution as the midpoints $B_1, \ldots, B_{50}$, and then use the midpoint rule to evaluate $\displaystyle \int_0^1 g_1(A, B) dB$.

\subsection{Standardized Gaussian kernel functions} \label{eba:gskfunctions}
\citet{Reich2012} use standardized Gaussian kernel functions as their spatial basis functions in the low-rank max-stable model.
Consider a set of $\bk_1, \ldots, \bk_L$ spatial knot locations in $\calD^2$, the region of interest.
Then
\begin{align}
  \hat{B}_{il} = \frac{\exp\left\{- \displaystyle\frac{||\bs_i - \bk_l||^2}{2 \rho^2} \right\}}{\displaystyle \sum_{j = 1}^L \exp\left\{- \frac{||\bs_i - \bk_j||^2}{2 \rho^2} \right\}}
\end{align}
where $|| \cdot ||$ is the Euclidean distance between a site and a knot location.

\subsection{Principal components for fire and precipitation data}
As a comparison to the EBFs, \fref{ebfig:fire-eigpanel} gives the first six principal components for the fire data, and \fref{ebfig:precip-eigpanel} gives the first six principal components for the precipitation data.
\begin{figure}[htbp] % markdown/fire-analysis/basis-functions.R
  \centering
  \includegraphics[width=\linewidth]{plots/fire-eig-panel.pdf}
  \caption{First six principal components for the Georgia fire data.}
  \label{ebfig:fire-eigpanel}
\end{figure}

\begin{figure}[htbp] % markdown/fire-analysis/basis-functions.R
  \centering
  \includegraphics[width=0.5\linewidth]{plots/firelambda-25.pdf}
  \caption{Cumulative sum of the first 25 eigenvalues.}
  \label{ebfig:fire-eigpanel}
\end{figure}

\begin{figure}[htbp]  % markdown/precipitation/cv-setup.R
  \centering
  \includegraphics[width=\linewidth]{plots/precip-eig-panel.pdf}
  \caption{First six principal components for the precipitation data.}
  \label{ebfig:precip-eigpanel}
\end{figure}

\begin{figure}[htbp]  % markdown/precipitation/cv-setup.R
  \centering
  \includegraphics[width=0.5\linewidth]{plots/preciplambda-25.pdf}
  \caption{Cumulative sum of the first 25 eigenvalues.}
  \label{ebfig:precip-eigpanel}
\end{figure}

One reason why the principal components for the fire data look so different from the EBFs presented in \fref{ebfig:fire-ebfpanel} is that the EBFs use censored data whereas the principal components do not.

\begin{singlespace}
\bibliographystyle{rss}
\bibliography{library}
\end{singlespace}

\end{document}




